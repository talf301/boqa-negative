\section{Previous Work}
\label{sec:lit-rev}

Bauer et al.\ \cite{bauer2012bayesian} introduced a disease diagnosis system
that uses Bayesian inference to resolve diagnostic queries, entitled the 
{\it Bayesian Ontology Query Algorithm} (BOQA).
%
The system comprises a Bayesian network with three layers of Boolean variables:
an {\it item} layer of diseases, a {\it hidden} layer of phenotypic features,
and a {\it query} layer of phenotypic features.
%
\footnote{
    In this report, we use the term {\it phenotype} interchangeably with symptom,
    since all symptoms of interest to the model are expected to have their root
    in genetic disorders.
}
%
The causal relationships between a disease and its related symptoms are modelled
as a set of directed edges from each disease node in the {\it item} layer to its
subset of phenotypic nodes in the {\it hidden} layer.
%
We may refer to the symptoms caused by a disease as the disease's {\it phenotype
annotations}.
%
Each phenotype node in the hidden layer is paired with exactly one node in the
query layer, by a directed edge from the node in the hidden layer to its
correspondent in the query layer.

We may interpret this model in a generative fashion as follows:
%
the occurrence of a disease (activity in an item node) causes its symptoms to
occur with some probability (activity in a hidden node). Furthermore, the
presence of a symptom (activity in a hidden node) may lead to the clinician
detecting the symptom (activity in a query node). However, the clinician may
fail to recognize the occurrence of a symptom (activity in a hidden node without
activity in the corresponding query node), or the clinician could perceive a
symptom where none exists (activity in a query node without
activity in the corresponding hidden node).
%
In this way, the hidden layer allows the system to model uncertainty about the
accuracy of a clinician's symptomatic description.

In addition to the edges between layers, BOQA contains within-layer connections
to model the ontology of phenotypes described by the Human Phenotype Ontology
(HPO) \cite{kohler2014hpo}.
%
Within the hidden layer, there is a directed edge from node $H_a$ to $H_b$
if the annotation of the phenotype $H_a$ to a disease implies that the phenotype
$H_b$ is also annotated to the disease.
%
Such an edge is present in the network if the phenotype $H_a$ is a child of
phenotype $H_b$ in the HPO. 
%
These edges serve the purpose of encoding the {\it annotation propagation rule}:
if an item $i$ is annotated to term $j$ then it is implicitly annotated to all
ancestors of $j$.

Lastly, in the case that the edge from $H_a$ to $H_b$ is present, then  in the
query layer, there is a directed edge from phenotype $Q_b$ to phenotype $Q_a$ (i.e., an edge in the
inverse direction of the edge $H_a \to H_b$).
%
These edges serve to encode the fact that a if a phenotype $Q_a$ is in the search query, then so are all of its ancestors.
%
\begin{center}
\begin{figure}[h]
\newcommand{\itemlayer}{$I_1$/A, $I_2$/B}
\newcommand{\hiddenquerylayers}{$H_1$/C/$Q_1$/J, $H_2$/D/$Q_2$/K, $H_3$/E/$Q_3$/L, $H_4$/F/$Q_4$/M, $H_5$/G/$Q_5$/N, $H_6$/H/$Q_6$/O, $H_7$/I/$Q_7$/P}
%
    \label{fig:bauer-net}
    \begin{tikzpicture}
        \pgfmathsetmacro\diametersqr{1}
        \pgfmathsetmacro\rectwidth{2}
        \pgfmathsetmacro\rectheight{8}
        \pgfmathsetmacro\spacing{1}
        
        \pgfmathsetmacro\numnodesitem{2}
        \pgfmathsetmacro\numnodeshiddenquery{7}
        
        % item layer
        \xdef\xlist{4}
        \xdef\ylist{4}
        \pgfmathsetmacro\yoffset{\rectheight / \numnodesitem}
        \foreach \m/\n [count=\l] in \itemlayer{
            \foreach \k in {1,...,400}{ % try 400 times to place without a collision
                \pgfmathsetmacro\x{rnd*\rectwidth}
                \pgfmathsetmacro\y{(\l - 1)*\yoffset + rnd*\yoffset}
                \xdef\collision{0}
                \foreach \element [count=\i] in \xlist{
                    \pgfmathtruncatemacro\j{\i-1}
                    \pgfmathsetmacro\checkdistancesqr{ ( ({\xlist}[\j]-(\x))^2 + ({\ylist}[\j]-(\y))^2 ) }
                    \ifdim\checkdistancesqr pt<\diametersqr pt
                        \xdef\collision{1}
                        \breakforeach
                    \fi
                }
                \ifnum\collision=0
                    \xdef\xlist{\xlist,\x}
                    \xdef\ylist{\ylist,\y}
                    \draw (\x,\y) node[item on] (\n) {\m};
                    \breakforeach
                \fi 
            }
        }
        
        % hidden layer
        \xdef\xlist{4}
        \xdef\ylist{4}
        \pgfmathsetmacro\yoffset{\rectheight / \numnodeshiddenquery}
        \foreach \m/\n/\o/\p [count=\l] in \hiddenquerylayers{
            \foreach \k in {1,...,400}{ % try 400 times to place without a collision
                \pgfmathsetmacro\x{((-1)^\l)*rnd*\rectwidth + 2*\rectwidth + \spacing}
                \pgfmathsetmacro\y{(\l - 1)*\yoffset + rnd*\yoffset}
                \xdef\collision{0}
                \foreach \element [count=\i] in \xlist{
                    \pgfmathtruncatemacro\j{\i-1}
                    \pgfmathsetmacro\checkdistancesqr{ ( ({\xlist}[\j]-(\x))^2 + ({\ylist}[\j]-(\y))^2 ) }
                    \ifdim\checkdistancesqr pt<\diametersqr pt
                        \xdef\collision{1}
                        \breakforeach
                    \fi
                }
                \ifnum\collision=0
                    \xdef\xlist{\xlist,\x}
                    \xdef\ylist{\ylist,\y}
                    \draw (\x,\y) node[hidden on] (\n) {\m};
                
                    \pgfmathsetmacro\x{\x + 2*\rectwidth + \spacing}
                    \draw (\x,\y) node[query on] (\p) {\o};
                    
                    \breakforeach
                \fi 
            }
        }
        
        % edges from item to hidden layer
        \edge {A} {C,F};
        \edge {B} {H};
        
        % edges from hidden to query layer
        \edge {C} {J};
        \edge {D} {K};
        \edge {E} {L};
        \edge {F} {M};
        \edge {G} {N};
        \edge {H} {O};
        \edge {I} {P};
        
        % edges within hidden layer
        \edge {C} {E};
        \edge {D} {E};
        \edge {E} {F};
        \edge {F} {G};
        \edge {G} {I};
        \edge {H} {I};
        
        % edges within query layer
        \edge {L} {J};
        \edge {L} {K};
        \edge {M} {L};
        \edge {N} {M};
        \edge {P} {N};
        \edge {P} {O};
        
        % surrounding boxes
        \node (X) [draw=blue, fit= (A) (B), inner sep=0.2cm, ultra thick, fill=blue!20, fill opacity=0.2] {};
        \node [yshift=2.0ex] at (X.north) {\textbf{item layer}};
        
        \node (Y) [draw=orange, fit= (C) (D) (E) (F) (G) (H) (I), inner sep=0.2cm, ultra thick, fill=orange!20, fill opacity=0.2] {};
        \node [yshift=2.0ex] at (Y.north) {\textbf{hidden layer}};
        
        \node (Z) [draw=purple, fit= (J) (K) (L) (M) (N) (O) (P), inner sep=0.2cm, ultra thick, fill=purple!20, fill opacity=0.2] {};
        \node [yshift=2.0ex] at (Z.north) {\textbf{query layer}};
    \end{tikzpicture}
    \caption{BOQA network structure.\footnotemark}
\end{figure}
\end{center}
\footnotetext{Reproduced from \bibentry{bauer2012bayesian}.}

\subsection{Network construction\footnotemark}
\label{subsec:netconst}
\footnotetext{Equations in \Section{subsec:netconst} are reproduced from \bibentry{bauer2012bayesian}.}

Let $M$ represent the number of terms in the ontology, and let $N$ represent the
number of diseases.
%
Let 
$\{I_i\}_{i=1}^{N}, 
\{H_j\}_{j=1}^{M}$ and 
$\{Q_k\}_{k=1}^{M}$ represent the nodes in the {\it item}, {\it hidden} and {\it
query} layers, respectively.

Indices of phenotypic nodes in this network correspond to indices of terms in
the HPO;
%
i.e., $H_i$ and $Q_i$ together correspond to the $i$th node in the ontology.
%
We identify parent-child relations from the ontology in the following manner:
%
let pa$(i) = \{\text{pa}(i)_1, \hdots, \text{pa}(i)_J\}$ denote the $J$ indices
of the direct parents of term $i$ in the ontology, 
%
and similarly let chi$(i) = \{\text{chi}(i)_1, \hdots, \text{chi}(i)_K\}$ 
denote the $K$ indices of the direct children of term $i$ in the ontology.

Furthermore, indices of disease nodes in the network correspond to indices of
diseases that are annotated by terms in the HPO.
%
Identify the annotations for a disease as follows: let ea$(j) =
\{\text{ea}(j)_1, \hdots, \text{ea}(j)_L\}$ denote the indices of the $L$ terms
for which the $j$th disease is explicitly annotated.

The local probability distributions for the hidden nodes can then be written as
%
\begin{align}
    P \left(H_i = 1 \mid I_{\text{ea}(i)}, \bigvee H_{\text{chi}(i)}\right)
    &= \left(
        1 - \prod_{j=\text{ea}(i)_1}^{\text{ea}(i)_L}
        \left(1 - I_j \, f_{ji}\right)
    \right)
    ^{1 - \bigvee H_{\text{chi}(i)}}
    \label{eq:lpdhids}
\end{align}
%
where $\bigvee$ represents the logical disjunction, and $f_{ji}$ represents the
empirical frequency of the occurrence of phenotype $i$ with disease $j$.
%
This formulation captures the annotation propagation rule within the hidden layer,
since it evaluates to $1$ if $\bigvee H_{\text{chi}(i)} = 1$; that is, if any
child annotations are active.
%
As well, it captures that the hidden node is inactive with probability 1 if all
diseases are inactive (i.e., $I_i = 0, \forall i$), and otherwise the
probability of activity in the hidden state is a function of the empirical
association of disease and symptom.

The local probability distribution for the query nodes, given the states of
$H_i$ and $Q_{\text{pa}(i)}$, can be written as
%
\begin{align*}
    \begin{aligned}[c]
        P\left(Q_i = 0 \mid H_i = 1, \bigwedge Q_{\text{pa}(i)} = 0\right) &= 1 \\
        P\left(Q_i = 1 \mid H_i = 1, \bigwedge Q_{\text{pa}(i)} = 0\right) &= 0 \\
        P\left(Q_i = 0 \mid H_i = 1, \bigwedge Q_{\text{pa}(i)} = 1\right) &= \beta \\
        P\left(Q_i = 1 \mid H_i = 1, \bigwedge Q_{\text{pa}(i)} = 1\right) &= 1 - \beta \\
    \end{aligned}
    \qquad
    \begin{aligned}[c]
        P\left(Q_i = 0 \mid H_i = 0, \bigwedge Q_{\text{pa}(i)} = 0\right) &= 1 \\
        P\left(Q_i = 1 \mid H_i = 0, \bigwedge Q_{\text{pa}(i)} = 0\right) &= 0 \\
        P\left(Q_i = 0 \mid H_i = 0, \bigwedge Q_{\text{pa}(i)} = 1\right) &= 1 - \alpha \\
        P\left(Q_i = 1 \mid H_i = 0, \bigwedge Q_{\text{pa}(i)} = 1\right) &= \alpha
    \end{aligned}
\end{align*}
%
where $\bigwedge$ represents the logical conjunction, $\beta$ represents the
probability of a false negative (i.e., $H_i = 1$ but $Q_i = 0$), and $\alpha$
represents the probability of a false positive (i.e., $Q_i = 1$ but $H_i = 0)$.

Now, defining a variable $m_{xyz\mid QH}$ by
%
\begin{align*}
    m_{xyz\mid QH} 
    &= \left| 
        \left\{k \mid
        \left(
            Q_k = x 
        \right) \wedge  \left(
            H_k = y
        \right) \wedge  \left(
            \bigwedge Q_{\text{pa}(k)} = z
        \right)
        \right\}
    \right|,
\end{align*}
%
the joint probability of the query nodes $Q_i$ may be written as
%
\begin{align*}
    \prod_{i=1}^M P\left(Q_i \mid H_i, \bigwedge Q_{\text{pa}(i)}\right)
    &=  \beta^{\,m_{011\mid QH}}
    \;(1 - \beta)^{m_{111\mid QH}}
    \;\alpha^{m_{001\mid QH}}
    \;(1 - \alpha)^{m_{101\mid QH}},
\end{align*}
under the assumption that the invalid configurations $m_{110\mid QH}$ and
$m_{100\mid QH}$ occur with probability zero,
%
and using the simplification that the configurations $m_{010\mid QH}$ and
$m_{000\mid QH}$ have probability one.

The joint probability distribution over the network is then realized as
%
\begin{align}
    &P(I_1, \hdots, I_N, H_1, \hdots, H_M, Q_1, \hdots, Q_M) \nonumber\\
    &=  P(I_1, \hdots, I_N) 
    \; \prod_{i=1}^M P\left(H_i \mid \bigvee I_{\text{ea}(i)}, \bigvee H_{\text{chi}(i)}\right)
    \; P\left(Q_i \mid H_i, \bigwedge Q_{\text{pa}(i)}\right). \label{eq:joint}
\end{align}

\subsection{Inference}

The marginal probability of a configuration of the disease nodes $I_1, \hdots,
I_N$ in the item layer, given some phenotypic evidence $Q_1, \hdots, Q_M$, is
given by marginalizing over $\vec{H}$ as
%
\begin{align*}
    P(I_1, \hdots, I_N \mid Q_1, \hdots, Q_M)
    &= \frac{\sum_{\vec{H}} P(I_1, \hdots, I_N, H_1, \hdots, H_M, Q_1, \hdots, Q_M)}{P(Q_1, \hdots, Q_M)} \\
    &= \frac{\sum_{\vec{H} \in \{0, 1\}^M} P(I_1, \hdots, I_N, H_1, \hdots, H_M, Q_1, \hdots, Q_M)}{P(Q_1, \hdots, Q_M)} \\
\end{align*}

The goal of this model is to find the configuration of diseases with highest
posterior probability, given some phenotypic evidence.
%
This is a MAP inference problem, and so may be solved by maximizing the product
of the likelihood and the prior; i.e., the solution is given by
%
\begin{align}
    &\argmax_{(I_1, \hdots, I_N)} \;
    P(I_1, \hdots, I_N \mid Q_1, \hdots, Q_M) \nonumber\\
    &= \argmax_{(I_1, \hdots, I_N)} \;
    \frac{P(Q_1, \hdots, Q_M \mid I_1, \hdots, I_N )\; P(I_1, \hdots, I_N)}{P(Q_1, \hdots, Q_M)} \nonumber\\
    &= \argmax_{(I_1, \hdots, I_N)} \;
    P(Q_1, \hdots, Q_M \mid I_1, \hdots, I_N )\; P(I_1, \hdots, I_N) \nonumber\\
    &= \argmax_{(I_1, \hdots, I_N)} \;
    P(I_1, \hdots, I_N)
    \;\sum_{\vec{H} \in \{0, 1\}^M} \; \prod_{i=1}^M \; P\left(H_i \mid \bigvee I_{\text{ea}(i)}, \bigvee H_{\text{chi}(i)}\right)
    \; P\left(Q_i \mid H_i, \bigwedge Q_{\text{pa}(i)}\right).\label{eq:mapinf}
\end{align}

\subsection{Complexity restrictions}

The computation in \Equation{eq:mapinf} is intractable, since 
there are $2^N$ configurations of the item nodes over which to maximize the
posterior, and there are $2^M$ configurations of the hidden nodes $H_1, \hdots,
H_M$ over which to marginalize.
%
Bauer et al.\ make two simplifications to make inference tractable; we describe
them in \Section{subsubsec:odc} and \Section{subsubsec:kleast}.

\subsubsection{One-disease constraint}
\label{subsubsec:odc}

To avoid maximizing over the $2^N$ configurations of the item nodes, Bauer et al.\
impose the constraint
%
\begin{align}
    \sum_{i=1}^N I_i = 1. \label{eq:onehot}
\end{align}
%
\Equation{eq:onehot} enforces the restriction that only one disease node may be
active given any query, and so the maximum is taken over $N$ one-hot
configurations, in order to compute the MAP estimate.
%
This is equivalent to assuming that a patient can have only a single disease;
%
since this is a diagnosis system for rare genetic diseases, this is a reasonable
assumption.
%
Therefore, we adopt the one-disease constraint into all models that we describe
in \Section{sec:models}.

\subsubsection{$k$-least frequency annotations constraint}
\label{subsubsec:kleast}

To avoid marginalizing over the $2^M$ configurations of the hidden nodes,
Bauer et al.\ simplify the local probability distribution of each hidden node
that is given by \Equation{eq:lpdhids}.
%
In particular, they restrict the number of frequency annotations $f_{ji}$, for
each disease node $I_i$, that are not exactly zero or exactly one, to the
annotations with the $k$ least frequency values.
%
\footnote{
    Bauer et al.\ use $k=10$ for their final experimental results.
}
%
This has the effect of enforcing all other hidden node likelihoods
to be either active or inactive with probability one,
conditional on a disease node.
%
Therefore, to compute the sum over in $\vec{H} \in \{0, 1\}^M$ in
\Equation{eq:mapinf}, the model need only take into account the subset of
summands representing configurations in which the likelihood of all hidden nodes
with deterministic activity, is one, since the remaining terms are zero.
%
This makes the marginalization computationally tractable.
%
In \Section{sec:models}, we describe models that either adopt or relax this
constraint.
