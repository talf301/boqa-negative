\section{Modifications}
\label{sec:models}

In this section we describe the various modifications to
the network structure and inference procedure made in order to incorporate the
desired functionality.
\footnote{
    Our code can be found at
    \url{https://github.com/talf301/boqa-negative}.
    We reimplemented the BOQA baseline codebase in Python, and manually coded all
    modifications described in this section.
}

\subsection{Modification to the network structure}
\label{subsec:modnetstruct}

Recall that in the BOQA network, the probability of activity in a hidden node is
realised as  
%
\begin{align}\label{eq:lpdhidden}
    P \left(H_i = 1 \mid I_{\text{ea}(i)}, \bigvee H_{\text{chi}(i)}\right)
    &= \left(
        1 - \prod_{j=\text{ea}(i)_1}^{\text{ea}(i)_L}
        \left(1 - I_j \, f_{ji}\right)
    \right)
    ^{1 - \bigvee H_{\text{chi}(i)}}
\end{align}
%
where $f_{ji}$ represents the empirical frequency of the occurrence of phenotype
$i$ with disease $j$.
%
Under this formulation, if disease $i$ is not annotated to phenotype $j$, 
and none of the children of $j$ are either (i.e., $\bigvee H_{\text{chi}(i)} = 0$),
then the probability of activity in the hidden layer is simply zero.
%
In other words, the likelihood of observing symptom $j$, given that the patient
has disease $i$, is zero.

However, since the symptom is not negatively annotated to the disease, we would
like the model instead to assign some likelihood to the occurrence of this event,
encoding the expectation that the symptom may occur together with the disease
by chance.
%
In particular, a likelihood of zero should be assigned only to those symptoms
that are negatively annotated to a disease.
%
We modify the local probability distribution in (\ref{eq:lpdhidden}) to capture
this specification as described in the following paragraphs.

Let $\text{pos}(i) = \{\text{pos}(i)_1, \hdots, \text{pos}(i)_S\}$, for each
$H_i$, index the $S$ diseases for which $H_i$ is explicitly positively
annotated and let $\text{neg}(i) = \{\text{neg}(i)_1, \hdots,
\text{neg}(i)_T\}$, the $T$ diseases for which $H_i$ is explicitly negatively
annotated.

Then if we let the probability of activity in a hidden node $H_j$ be given by
\begin{align}\label{eq:lpdhiddenmod1}
    &P\left(H_i = 1 \mid I_{\text{pos}(i)_1}, \hdots, I_{\text{pos}(i)_S},
    \bigvee I_{\text{neg}(i)}, \bigwedge H_{\text{chi}(i)}\right)\nonumber\\
        &= \left(
            1 - 
            \left(
                \prod_{j=\text{pos}(i)_1}^{\text{pos}(i)_S}
                \left(1 - I_j \, f_{ji}\right)
            \right) ^{1 - \bigvee I_{\text{neg}(i)}}
        \right)
        ^{1 - \bigvee H_{\text{chi}(i)}}
\end{align}
where if the empirical frequency is not available but the symptom $j$ is not
negatively annotated to disease $i$, then frequency $f_{ji}$ is set to some
small value, $p$.
%
\footnote{It should be noted that activity of a child annotation of a node
    $H_j$ entails activity in $H_j$ under the formulation in
    (\ref{eq:lpdhiddenmod1}), even if phenotype $j$ is negatively annotated
    to the disease of interest. However, this situation does not occur in the
    HPO, and so we need not treat special cases.
}

However, the result of this modification to the network is that exact inference
is now intractable, since marginalising over all binary assignments to the
hidden nodes is exponential in the number of phenotypes,
as noted in \Section{subsubsec:kleast}. 
%
Furthermore, we may not apply the $k$-least frequencies restriction 
described in \Section{subsubsec:kleast}. 
to simplify the marginalisation, since that would force some hidden nodes to
deterministically be inactive, conditional on activity in
some item node, even though the corresponding phenotypes  may not be negatively
annotated to the active disease.
%
Therefore, we consider several methods to approximate computation of the
marginals, which we describe in the next sections.

\subsection{Modifications to the inference procedure}
\label{subsec:modinf}

We test two different methods of sampling to approximate inference in the
network.

\subsubsection{$p$-sampling}
\label{subsubsec:psampmodel}
%
As described for \Equation{eq:lpdhiddenmod1}, we
we assume that for each hidden node without a negative annotation, the frequency
has a fixed value of $p$. 
%
More details are given in \Section{subsubsec:psampexp}.

\subsubsection{Information-content sensitive $p$-sampling}
%
Here, Rather than using a single value $p$ as frequency, we weight these
probabilities by the inverse information content of the phenotype
associated with the hidden node. \Equation{eq:infocontent} shows how
we compute the information content for each phenotype.
%
\begin{align}\label{eq:infocontent} 
    -\log_{2}\frac{\text{\# of annotations for disease } i \text{ including
    descendants}}{N},
\end{align}
where $N$ is the number of known diseases.
%
More details are given in \Section{subsubsec:icpsampexp}.
