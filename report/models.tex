\section{Models}
\label{sec:models}

In this section we describe the models resulting from various modifications to
the network, in order to incorporate the desired functionality.

Recall that in the BOQA network, the probability of activity in a hidden node is
realised as  
%
\begin{align}\label{eq:lpdhidden}
    P \left(H_i = 1 \mid I_{\text{ea}(i)}, \bigvee H_{\text{chi}(i)}\right)
    &= \left(
        1 - \prod_{j=\text{ea}(i)_1}^{\text{ea}(i)_L}
        \left(1 - I_j \, f_{ji}\right)
    \right)
    ^{1 - \bigvee H_{\text{chi}(i)}}
\end{align}
%
where $f_{ji}$ represents the empirical frequency of the occurrence of phenotype
$i$ with disease $j$.
%
Under this formulation, if disease $i$ is not annotated to phenotype $j$, 
and none of the children of $j$ are either (i.e., $\bigvee H_{\text{chi}(i)} = 0$),
then the probability of activity in the hidden layer is simply zero.
%
In other words, the likelihood of observing this symptom, given that the patient
has disease $i$, is zero.

However, since the symptom is not negatively annotated to the disease, we would
like the model instead to assign some likelihood to the occurrence of this event,
encoding the expectation that the symptom may occur together with the disease
by chance.
%
In particular, a likelihood of zero should be assigned only to those symptoms
that are negatively annotated to the disease.
%
We modify the local probability distribution in (\ref{eq:lpdhidden}) to capture
this specification as described in the following paragraphs.

Let $\text{pos}(i) = \{\text{pos}(i)_1, \hdots, \text{pos}(i)_S\}$, for each
$H_i$, index the $S$ diseases for which $H_i$ is explicitly positively
annotated and let $\text{neg}(i) = \{\text{neg}(i)_1, \hdots,
\text{neg}(i)_T\}$, the $T$ diseases for which $H_i$ is explicitly negatively
annotated.

Then if we let the probability of activity in a hidden node $H_j$ be given by
\begin{align}\label{eq:lpdhiddenmod1}
    &P\left(H_i = 1 \mid I_{\text{pos}(i)_1}, \hdots, I_{\text{pos}(i)_S},
    \bigvee I_{\text{neg}(i)}, \bigwedge H_{\text{chi}(i)}\right)\nonumber\\
        &= \left(
            1 - 
            \left(
                \prod_{j=\text{pos}(i)_1}^{\text{pos}(i)_S}
                \left(1 - I_j \, f_{ji}\right)
            \right) ^{1 - \bigvee I_{\text{neg}(i)}}
        \right)
        ^{1 - \bigvee H_{\text{chi}(i)}}
\end{align}
where if the empirical frequency is not available but the symptom $j$ is not
negatively annotated to disease $i$, then frequency $f_{ji}$ is set to some
small value, $p$.
%
\footnote{It should be noted that activity of a child annotation of a node
    $H_j$ entails activity in $H_j$ under the formulation in
    (\ref{eq:lpdhiddenmod1}), even if phenotype $j$ is negatively annotated
    to the disease of interest. However, this situation does not occur in the
    HPO, and so we need not treat special cases.
}
%

However, the result of this modification to the network is that exact inference
is now intractable, since marginalising over all binary assignments to the
hidden nodes is exponential in the number of phenotypes (i.e., there are $2^M$
configurations to consider). Therefore, we consider several methods to
approximate computation of the marginals.

\subsection{Sampling}
In the original model, rather than using the method of \Section{subsubsec:kleast}, we
apply a monte carlo approach. For each item node, we independently sample each edge
which has frequency information, creating n=1000 models. We then compute the marginals
over each of these models, and take an average to get our final marginals. This process
of sampling edges based on frequencies is equivalent to probabilistically determining
the activity of the hidden layer based on frequency information.
\subsection{$p$-sampling}
Since our modification to the model makes exact inference intractable, we again apply
a monte carlo approach. In this model, we apply the same sampling for edges with frequency
information, but additionally we assume that there is an edge to each hidden node without
a negative annotation, with frequency $p$. For computational reasons, we precompute the
same sample of edges with probability $p$ for all item nodes, and then apply necessary per-node
transformations.
\subsection{Information-content sensitive $p$-sampling}
Rather than using a single value $p$ as the probability for any edge to exist, we weight these
probabilities inversely based on the information content of the phenotype associated with the hidden node. 
Equation (\ref{eq:infocontent}) describes how we compute the information content for each phenotype.
\begin{align}\label{eq:infocontent}
	-\log_{2}\frac{\text{\# of annotations for disease} i \text{including descendants}}{N}
\end{align}



\footnote{Link to repo:
    \url{https://github.com/talf301/boqa-negative}
}

