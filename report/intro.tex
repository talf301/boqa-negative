\section{Introduction}
\label{sec:intro}

Computer-Assisted Diagnosis (CAD) is a useful and desirable tool that has been challenging the AI community for many years. For example, Jaakkola et al.\ \cite{qmrdt} introduced the QMR-DT network, a noisy-\texttt{OR} probabilistic graphical model, constructed from expert knowledge.
In particular, CAD is useful in the realm of rare genetic diseases, since clinical geneticists will often be tasked with diagnosing patients that have disorders they may only see a couple of times in their careers. In this regard, having a tool which can make reasonable suggestions for diagnoses a clinician may have never seen before, is invaluable.

In order for a CAD system to be feasible, one requires a standardized dictionary for symptoms and diseases, as well as either a very large training set of diagnosed patients, or some expert information relating symptoms and diseases. While the first option is not possible due to the rarity of the diseases being investigated, the Human Phenotype Ontology (HPO) \cite{kohler2014hpo} and Online Mendelian Inheritance in Man (OMIM) \cite{omim} together provide the standardized dictionary, as well as the necessary expert information. Specifically, the HPO contains information, for each disease, about empirical frequencies with which related symptoms appear, as well as symptoms that will never appear; we call the latter {\it negative annotations}.

These negative annotations are conceptually difficult to incorporate into a model, and because of the minor role they play in diagnosis, will typically be ignored in CAD systems. In this work, we attempt to develop a model that makes use of these negative annotations in order to improve CAD.