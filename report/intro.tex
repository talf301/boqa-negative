\section{Introduction}
\label{sec:intro}

[TODO: introduce disease diagnosis in general, reference QR network or something like that]
Computer assisted diagnosis is a useful and desirable tool that has been challenging the AI community for many years. Jaakkola et Al. \cite{qmrdt} introduced the QMR-DT network, an example of a major use of a probabilistic graphical model built on expert knowledge used for CAD. 
In particular, computer-assisted diagnosis (CAD) is useful in the realm of rare genetic diseases: clinical geneticists will often be tasked with diagnosing patients with disorders they may only see a couple of times in their careers. In this regard, having a tool which can make some reasonable suggestions for diagnoses a clinician may have never seen before is invaluable. \\
In order for a CAD system to be feasible, one requires either a very large training set of diagnosed patients, or some expert information relating disorders and symptoms. Clearly, the first option is not a possibility in our domain, but the Human Phenotype Ontology (HPO) \cite{kohler2014hpo} and OMIM \cite{omim}

% TODO: emphasise that we are talking about rare genetic diseases and not other types of diseases