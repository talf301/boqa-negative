\section{Experiments}
\label{sec:exp}

\subsection{Data \& data generation}

The HPO \cite{kohler2014hpo} is a directed acyclic graph that depicts an
ontology of 10 476 phenotypes.
%
It specifies annotations for 6575 diseases, 471 of which possess negative
annotations.
%
In addition to using the ontology as expert information in the model
construction and inference procedures, we exploit it to generate patients
for testing.
%
\footnote{
    Rare genetic diseases is, true to its name, a domain in which there are few
    examples, so we have to generate more test data to fill the void.
}
%
We spawned 100 patients infected with diseases that
possess positive annotations but not necessarily negative annotations, and then
generated a further 100 patients each infected with a
negatively-annotated disease.

We generated and infected patients in the following manner:
%
\begin{enumerate}
    \item \label{enum:patgen1}
        For a given disease, we sampled each of the disease's annotated
        symptoms with probability equal to the empirical association of symptom
        and disease, as specified by the HPO annotations.
    \item \label{enum:patgen2}
        To simulate noise in the patient queries, we added a number
        of unrelated symptoms to the patient query by sampling with uniform
        probability over all symptoms.
        \footnote{
            For all experimental results, we chose the number of noise symptoms
            to be half the number of symptoms generated in Step
            \ref{enum:patgen1}.
        }
    \item To simulate imprecision in the patient queries, for each symptom
        generated in Steps \ref{enum:patgen1} and \ref{enum:patgen2}, with some
        probability we replaced the symptom with one of its ancestors in the
        HPO.
        \footnote{
            Each symptom is replaced by another symptom chosen uniformly from a
            set containing all ancestors of the symptom and the symptom itself.
            In this way, there is some small probability that the symptom is
            retained, and thus the query is made no less precise.
        }
\end{enumerate}
%
We will refer to the data generated in this way as the {\it artificial patient
data}.

In addition to this artificially generated data, we tested all models
on 101 patient queries obtained from real-life clinician-patient
encounters, which we refer to as the {\it naturalistic patient data}.
%
\footnote{
    We take the naturalistic patients data from the PhenoTips patients
    repository \cite{phenotips}.
    The data is not publicly available.
}

\subsection{Models to test}
%
In order to test our modifications as outlined in \Section{subsec:modnetstruct},
we performed experiments on various models which differ in their structure and
inference methods.
%
\Section{subsubsec:nofreqmodel} and 
\Section{subsubsec:kleastmodel} describe our benchmark, and
\Section{subsubsec:sampmodel}, 
\Section{subsubsec:psampexp} and 
\Section{subsubsec:icpsampexp} describe the models introduced by this report.

\subsubsection{No frequency annotations model}
\label{subsubsec:nofreqmodel}
%
This model is a simplification of the model of Bauer et al.\
\cite{bauer2012bayesian}; specifically, all stochastic state propagations are
made deterministic; in other words, it is assumed that the empirical frequency
in 
\Equation{eq:lpdhids} is either zero or one.
%
Exact inference is performed.
%
We use this model to benchmark our results.

\subsubsection{$k$-least frequency annotations model}
\label{subsubsec:kleastmodel}
%
This model is identical to the final model of Bauer et al.\
\cite{bauer2012bayesian}. Specifically, the number of non-deterministic activity
state propagations from the item layer to the hidden layer is capped at $k$ for
each item node, and exact inference is performed.
%
We use this model as well to benchmark our results.

\subsubsection{Sampling model}
\label{subsubsec:sampmodel}
%
We take the model of Bauer et al.\ \cite{bauer2012bayesian} with all frequency
annotations; however, we do not perform exact inference.
%
Instead, for each item node, we independently sample each edge that has frequency
information, with probability equal to the frequency, creating $n$\footnote{
    For $p$-sampling and information-content sensitive $p$-sampling, we chose $n=1000$.
} instances of the network.
%
We then compute the item node marginals for each of these models, and take
the average as the value of the posterior probability of each disease. 

This process of sampling edges based on frequencies is equivalent to
fixing the stochastic activity of the hidden layer based on
frequency information;
%
therefore we are sampling over models that have a fixed activity configuration in the
hidden layer.
%
In short, this is Bauer et al.'s model without the constraint in
\Section{subsubsec:kleast}, and with approximated inference for the
marginals, accomplished by sampling.

\subsubsection{$p$-sampling model with negative annotations}
\label{subsubsec:psampexp}
%
This model is the BOQA model, with the modified hidden layer local probability
distribution described in \Equation{eq:lpdhiddenmod1}, and the specification for the
sampling procedure described in \Section{subsubsec:psampmodel},
with the sampling itself being done as described in \Section{subsubsec:sampmodel}.
%
It therefore utilizes negative annotations.

\subsubsection{Information-content sensitive $p$-sampling model with negative
annotations}
\label{subsubsec:icpsampexp}
%
Again, this is the BOQA model with the modified hidden layer local probability
distribution described in \Equation{eq:lpdhiddenmod1}. The specification for the sampling
procedure used is that described in \Section{subsubsec:icsampmodel},
with the sampling itself being done as described in \Section{subsubsec:sampmodel}.
%
It also therefore utilizes negative annotations.
