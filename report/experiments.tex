\section{Experiments}
\label{sec:exp}

\subsection{Artificial data generation}

The HPO \cite{kohler2014hpo} is a directed acyclic graph that depicts an
ontology of 10 476 phenotypes.
%
It specifies annotations for 6575 diseases, 471 of which possess negative
annotations.
%
In addition to using the ontology as expert information in the model
construction and inference procedures, we exploit it to generate patients
for testing.
%
\footnote{
    Rare genetic diseases is, true to its name, a domain in which there are few
    examples, so we have to generate more test data to fill the void.
}
%
We spawned 500 patients infected with diseases that
possess positive annotations but not necessarily negative annotations, and then
generated a further 471 patients each infected with a unique
negatively-annotated disease.

We generated and infected patients in the following manner:
%
\begin{enumerate}
    \item \label{enum:patgen1}
        For a given disease, we sampled each of the disease's annotated
        symptoms with probability equal to the empirical association of symptom
        and disease, as specified by the HPO.
    \item \label{enum:patgen2}
        To simulate noise in the patient queries, we added a number
        of unrelated symptoms to the patient query by sampling with uniform
        probability over all symptoms.
        \footnote{
            For all experimental results, we chose the number of noise symptoms
            to be half the number of symptoms generated in Step
            \ref{enum:patgen1}.
        }
    \item To simulate imprecision in the patient queries, for each symptom
        generated in Steps \ref{enum:patgen1} and \ref{enum:patgen2}, with some
        probability we replaced the symptom with one of its ancestors in the
        HPO.
        \footnote{
            Each symptom is replaced by another symptom chosen uniformly from a
            set containing all ancestors of the symptom and the symptom itself.
            In this way, there is some small probability that the symptom is
            retained, and thus the query is made no less precise.
        }
\end{enumerate}
%
We will refer to the data generated in this way as the {\it artificial patient
data}.

In addition to this artificially generated data, we tested all models
on 101 patient queries obtained from real-life clinician-patient
encounters, which we refer to as the {\it naturalistic patient data}.
%
\footnote{
    We take the naturalistic patients data from the PhenoTips patients
    repository \cite{phenotips}.
    The data is not publicly available.
}



