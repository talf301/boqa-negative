\section{Objectives}
\label{sec:obj}

The main contribution of this report is to incorporate the negative phenotype
annotations given by OMIM \cite{hamosh2005online} into
\citeauthor{kohler2014hpo}'s BOQA network.
By doing this, we wish to improve classification accuracy for diseases that
have negative annotations, while not diminishing performance for cases in which
the disease is only positively annotated.
As well, we require that the complexity we add to the model does not cause the
runtime of the algorithm to become intractable.

We evaluate the augmented model by
precision/recall and receiver operating characteristic (ROC) curves,
as well as mean reciprocal rank.

WE MAY TAKE THE FOLLOWING OUT IF NO TIME

Also, since the structure of the network is such that the local conditional
probability distributions are directly interpretable, we make and test
conjectures about the posterior probabilities of each disease. 
The following paragraphs summarise our predictions.

Consider two diseases, $I_1$ and $I_2$, that have identical phenotype
annotations, except a single symptom $Q$ that is negatively annotated to $I_1$.
In order for the system to make the best diagnosis, if a patient exhibits all
symptoms annotated to both diseases, yet does not exhibit phenotype $Q$, we
require that the system assign higher posterior probability to $I_1$.
Furthermore, we need the system to assign higher posterior probability to $I_2$
if the patient displays phenotype $Q$.

Now, consider a single disease $I$ and two patients who share the same symptoms
except for a phenotype, $Q$, for which $I$ is negatively annotated. We require that
the system assign higher posterior probability to the patient who does not
exhibit phenotype $Q$.

% finish writing; there are six cases

Formally:

Let $(Q_1, ..., Q_M)$ be query variables.
Let $I_1$ be a disease that is explicitly negatively annotated to $Q_1$.
Let $I_2$ be a disease that is not explicitly annotated to $Q_1$.
Let $I_3$ be a disease that is explicitly annotated to $Q_1$.

\begin{align*}
    P(Q_1 = 0, \hdots, Q_M \mid I_1) \\
    P(Q_1 = 1, \hdots, Q_M \mid I_3) \\
    P(Q_1 = 0, \hdots, Q_M \mid I_2) \\
    P(Q_1 = 1, \hdots, Q_M \mid I_2) \\
    P(Q_1 = 0, \hdots, Q_M \mid I_3) \\
    P(Q_1 = 1, \hdots, Q_M \mid I_1) 
\end{align*}

