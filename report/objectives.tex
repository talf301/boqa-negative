\section{Objectives}
\label{sec:obj}

The main contribution of this report is to incorporate the negative phenotype
annotations given by OMIM \cite{hamosh2005online} into
Bauer et al.'s \cite{bauer2012bayesian} BOQA network.
By doing this, we wish to improve classification accuracy for diseases that
have negative annotations, while not diminishing performance for cases in which
the disease is only positively annotated.
As well, we require that the complexity we add to the model does not cause the
runtime of the algorithm to become intractable.
%
Also, since the structure of the network is such that the local conditional
probability distributions are directly interpretable, we can make interpretable
requirements about the posterior probabilities of each disease. 
The following paragraphs summaries the requirements.

Consider two diseases, $I_1$ and $I_2$, that have identical phenotype
annotations, except a single symptom $Q_1$ that is negatively annotated to $I_1$.
In order for the system to make the best diagnosis, if a patient exhibits all
symptoms annotated to both diseases, yet does not exhibit phenotype $Q_1$, we
require that the system assign higher posterior probability to $I_1$.
Furthermore, we need the system to assign higher posterior probability to $I_2$
if the patient displays phenotype $Q_1$.

Now, consider a single disease $I_1$ and two patients who share the same symptoms
except for a phenotype, $Q_1$, for which $I_1$ is negatively annotated. We require that
the system assign higher posterior probability to the patient who does not
exhibit phenotype $Q_1$.

Formally:
%
Let $(Q_1, ..., Q_M)$ be query variables.
Furthermore,
let $I_1$ be a disease that is explicitly negatively annotated to $Q_1$, and
let $I_2$ be a disease that is explicitly positively annotated to $Q_1$.

Then, we require that the following relationships hold:

\vspace{0.1cm}
\begin{align*}
    P(I_1 \mid Q_1 = 0, \hdots, Q_M) >
    P(I_2 \mid Q_1 = 0, \hdots, Q_M); \\\\
    P(I_1 \mid Q_1 = 0, \hdots, Q_M) >
    P(I_1 \mid Q_1 = 1, \hdots, Q_M).
\end{align*}
\vspace{0.1cm}

In \Section{sec:models}, we formulate our model so as to satisfy these requirements.

